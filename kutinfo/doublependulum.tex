\documentclass{article}
\usepackage{ifxetex,ifluatex}
\newif\ifxetexorluatex
\ifxetex
  \xetexorluatextrue
\else
  \ifluatex
    \xetexorluatextrue
  \else
    \xetexorluatexfalse
  \fi
\fi

\ifxetexorluatex
  \usepackage{fontspec}
\else
  \usepackage[T1]{fontenc}
  \usepackage[utf8]{inputenc}
  \usepackage{lmodern}
\fi

\usepackage{hyperref}
\usepackage{amsmath}

\title{Kettősinga szimuláció felhasználói dokumentációja}
\author{Nógrádi Zsófia}

\begin{document}
\maketitle
\pagebreak

\section{A feladat}
A kettősinga kaotikus mozgásának modellezésére íródott a program. A kettős inga két inga egymáshoz csatolásával keletkező egyszerű 
rendszer, melynek mozgását a kezdeti feltételek erősen befolyásolják. Ez az objektum a legelterjedtebb demonstrációs erszköze a 
kaotikus mozgásnak.

\section{A fizikai leírása}
Az alap fizikai ismereteinket használjuk fel arra, hogy felírjuk a Newton-féle erőtörtvényeket a rendszerre. A továbbiakban használjuk
a következő jelöléseket: 

\begin{align*}
x &:  az ingatömeg vízszintes helyzete \\
y &: az ingatömeg függőleges helyzete \\
\theta &: az inga szöge \\
L &: a kötél hossza \\
F &: a kötélerő \\
m &: az ingatömege \\
g &: nehézségi gyorsulás értéke
\end{align*}

Pusztán geometriai tudáunkat elővéve felírjuk az ingatömegek helyzetét: 
\begin{gather*}
x_1=L_1 \sin \theta_1 \\
y_1= - L_1 \cos \theta_1 \\
x_2 = x_1 + L_2 \sin \theta_2 \\
y_2= y_1 - L_2 \cos \theta_2 \\
\end{gather*}

Ezeket az egyenleteket kétszer deriválva megkapjuk a gyorsulás értékét, majd azt felhasználva felírhatjuk a dinamika alaptörvényét, 
ami alapján $F=m \cdot a$ Az így kapott egyenleteket $\theta ''$-re rendezve adódnak azon differenciálegyenletek, amiket a programunk 
Runge-Kutta-módszerrel oldja meg. A továbbiakban elsősorban jelölés szempontjából bevezetjük a szögsebességet $\omega $ ami éppen 
$\theta$ első deriváltja. Így a Runge-Kuttának már négy elsőrendű differenciálegyenletet tudunk átadni, amit az meg tud oldani. Ezek 
a következőek:

\begin{gather*}
\theta_1' = \omega_1 \\
\theta_2' = \omega_2 \\
\omega_1' = \frac{-g(2m_1+m_2)\sin \theta_1 - m_2\cdot g \cdot \sin(\theta_1-2\theta_2) - 2 \sin(\theta_1 - \theta_2)
m_2(\omega_2^2 L_2 + \omega_1^2 L_1 \cos(\theta_1 - \theta_2))}{L_1(2m_1 + m_2 - m_2 \cdot \cos(2\theta_1 - 2 \theta_2))} \\
\omega_2'=\frac{2 \sin(\theta_1 - \theta_2)(\omega_1^2 \cdot L_1(m_1+m_2) + g(m_1+m_2) \cos \theta_1 + \omega_2^2 \cdot L_2 
\cdot m_2 \cos(\theta_1-\theta_2))}{L_2(2m_1+m_2-m_2\cos(2\theta_1 - 2 \theta_2))} \\
\end{gather*}

\section{Negyedrendű Runge-Kutta-módszer}
A módszer differenciálegyenletrendszerek megoldására alkalmas. Ahogy azt a neve is sugallja, ez egy negyedrendű közelítő módszer, 
ahol egy választott pici lépésközhöz n-edik (ahol n tipikus nagy) lépésben oldja meg a kezdetiértékproblémát.

\section{A program fontosabb paraméterei}
A program C++ program nyelven íródott, ami jól fordul CodeBlocks-ban Cmake, vagy C/C++ fordítással, és VisualStudioban is. Az
eredményeket a módszer segítségével gyorsan adja meg és írja ki.

\section{A program használata}
Két fontos részből épül fel a projekt. Az első fájl egy header fájl ("rungekutta.h"), ebben találhatók az inga adatai, amelyek a 
program elején vannak megadva, kommentelve, hogy melyik érték, mit jelöl a programban (a jelölések konzekvensek ezzel a segédlettel). 
Majd a következőben definiált "derivs" függvényben szerepelnek a kettősingára jellemző differenciálegyenletek, amelyeket a következőben 
definiált rungekutta függvény old meg a módszernek megfelelően, vagyis negyedrendű közelítéssel.
\newline
A program main.cpp-ben fut le, ahova be van hívva a korábban említett "rungekutta.h" header is. Itt megadjuk a lépésközöt, a kezdeti 
és a végső időpillanatot illetve a $\theta$ és $\omega$ kezdetiértékei is itt vannak definiálva. A program ezután allocálja a kellő 
memóriát az adatoknak, majd elvégzi a kitérésben szereplő szögekre való átírást, hogy a továbbiakban ezek fokokban legyen. Ezután jön
a rungekutta futtatása és a kapott paraméterek kiírása.

\end{document}
